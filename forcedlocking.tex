\documentclass[a4paper,14pt]{extarticle}

\usepackage{cmap}
\usepackage[T2A]{fontenc}
\usepackage[utf8x]{inputenc}
\usepackage[english, russian]{babel}

\usepackage{misccorr} % в заголовках появляется точка, но при ссылке на них ее нет
\usepackage{amssymb,amsfonts,amsmath,amsthm}  
\usepackage{indentfirst}
\usepackage[usenames,dvipsnames]{color} 
\usepackage[unicode,hidelinks]{hyperref}
% \hypersetup{%
%     pdfborder = {0 0 0}
% }

\usepackage{makecell,multirow} 
\usepackage{ulem}
\usepackage{graphicx,wrapfig}
\graphicspath{{img/}}
\usepackage{geometry}
\geometry{left=2cm,right=2cm,top=3cm,bottom=3cm,bindingoffset=0cm,headheight=15pt}
\usepackage{fancyhdr} 
\linespread{1.05} 
\frenchspacing 
\renewcommand{\labelenumii}{\theenumii)} 
\newcommand{\mean}[1]{\langle#1\rangle}
% \usepackage{caption}
%%%%%%%%%%%%%%%%%%%%%%%%%%%%%%%%%%%%%%%%%%%%%%%%%%%%%%%%%%%%%%%%%%%%%%%%%%%%%%%
%%%%%%%%%%%%%%%%%%%%%%%%%%%%%%%%%%%%%%%%%%%%%%%%%%%%%%%%%%%%%%%%%%%%%%%%%%%%%%%

\def\labauthors{Сарафанов Ф.Г., Платонова М.В.}
\def\labgroup{430}
% \def\department{Кафедра электроники и квантовой физики}
\def\labnumber{1}
% \def\labtheme{Измерение ширины запрещённой зоны в полупроводниках} 

%%%%%%%%%%%%%%%%%%%%%%%%%%%%%%%%%%%%%%%%%%%%%%%%%%%%%%%%%%%%%%%%%%%%%%%%%%%%%%%
	%применим колонтитул к стилю страницы
\pagestyle{fancy} 
	%очистим "шапку" страницы
\fancyhead{} 
	%слева сверху на четных и справа на нечетных
\fancyhead[L]{\labauthors} 
	%справа сверху на четных и слева на нечетных
% \fancyhead[R]{Отчёт по лабораторной работе №\labnumber} 
\fancyhead[R]{Вынужденная синхронизация} 
	%очистим "подвал" страницы
\fancyfoot{} 
	% номер страницы в нижнем колинтуле в центре
\fancyfoot[C]{\thepage} 
\renewcommand{\phi}{\varphi}
%%%%%%%%%%%%%%%%%%%%%%%%%%%%%%%%%%%%%%%%%%%%%%%%%%%%%%%%%%%%%%%%%%%%%%%%%%%%%%%

\usepackage{float}
\usepackage[mode=buildnew]{standalone}
\usepackage{tikz} 
% \usepackage{subcaption}
\usepackage{tikz,csvsimple}
\usetikzlibrary{scopes}
\usetikzlibrary{%
     decorations.pathreplacing,%
     decorations.pathmorphing,%
    patterns,%
    calc,%
    scopes,%
    arrows,%
    % arrows.spaced,%
}
\makeatletter
\newif\if@gather@prefix 
\preto\place@tag@gather{% 
  \if@gather@prefix\iftagsleft@ 
    \kern-\gdisplaywidth@ 
    \rlap{\gather@prefix}% 
    \kern\gdisplaywidth@ 
  \fi\fi 
} 
\appto\place@tag@gather{% 
  \if@gather@prefix\iftagsleft@\else 
    \kern-\displaywidth 
    \rlap{\gather@prefix}% 
    \kern\displaywidth 
  \fi\fi 
  \global\@gather@prefixfalse 
} 
\preto\place@tag{% 
  \if@gather@prefix\iftagsleft@ 
    \kern-\gdisplaywidth@ 
    \rlap{\gather@prefix}% 
    \kern\displaywidth@ 
  \fi\fi 
} 
\appto\place@tag{% 
  \if@gather@prefix\iftagsleft@\else 
    \kern-\displaywidth 
    \rlap{\gather@prefix}% 
    \kern\displaywidth 
  \fi\fi 
  \global\@gather@prefixfalse 
} 
\newcommand*{\beforetext}[1]{% 
  \ifmeasuring@\else
  \gdef\gather@prefix{#1}% 
  \global\@gather@prefixtrue 
  \fi
} 
\makeatother

\usepackage{booktabs}
\usepackage{pgfplots, pgfplotstable}

\usepackage[outline]{contour}
\usepackage{tocloft}
\renewcommand{\cftsecleader}{\cftdotfill{\cftdotsep}} % for parts
% \renewcommand{\cftchapleader}{\cftdotfill{\cftdotsep}} % for chapters
\usepackage{pgfplots,pgfplotstable,booktabs,colortbl}
\pgfplotsset{compat=newest}
\usepackage{physics}
\usepackage{mathtools}
\mathtoolsset{showonlyrefs=true}
\newcommand\Smat{\hat { \mathbf { S } }}

\newcommand*\dotvec[1][1,1]{\crossproducttemp#1\relax}
\def\crossproducttemp#1,#2\relax{{\qty[\vec{#1}\times\vec{#2}\,]}}

\newcommand*\prodvec[1][1,1]{\crossproducttempa#1\relax}
\def\crossproducttempa#1,#2\relax{{\qty[{#1}\times{#2}\,]}}

% \def\E{\mathscr{E}_H}
\def\Rdim{\,\frac{\text{м}^3}{\text{А} \cdot \text{с}}}

\renewcommand{\vec}{\mathbf} % for parts

\begin{document}
\begin{titlepage}
\begin{center}
% \vspace{-3em}
{\small\textsc{Нижегородский государственный университет имени Н.\,И. Лобачевского}}
\vskip 2pt \hrule \vskip 3pt
{\small\textsc{Радиофизический факультет}}

\vfill


{{\large Отчет по лабораторной работе №\labnumber}\vskip 12pt {\Huge \bfseries Вынужденная \\[10pt] синхронизация}}

	
\vspace{2cm}
{\large Работу выполнили студенты \\[-0.25em] 430 группы радиофизического факультата \\[0.5em] {\Large \bfseries \labauthors}}

% \vspace{0.5cm}
% {e-mail: sfg180@yandex.ru}

% \vspace{2cm}

\end{center}

\vfill
	
% \begin{flushright}
% 	{Выполнили студенты 430 группы\\ \labauthor}%\vskip 12pt Принял:\\ Менсов С.\,Н.}
% \end{flushright}
	
% \vfill
	
\begin{center}
	{Нижний Новгород, 23 апреля -- \today}
\end{center}

\end{titlepage}
\tableofcontents
\newpage




% \end{document}

\section*{Введение}
\addcontentsline{toc}{section}{Введение}

Синхронизация колебаний - согласование частот, фаз или др. характеристик сигналов, генерируемых взаимодействующими колебательными системами. Различают взаимную синхронизацию колебаний, когда парциальные подсистемы перестраивают режим колебаний друг друга, и внешнюю (вынужденную) синхронизацию колебаний, когда характеристики колебаний системы (систем) изменяются под действием внешней силы. Вынужденную синхронизацию по частоте колебаний называют захватыванием частоты.

Захватывание частоты - явление, состоящее в том, что автоколебательная система (автогенератор) при воздействии на неё периодически изменяющейся во времени внешней силы совершает колебания не с частотой автоколебаний и0, а с часто той и внешнего воздействия. Захват частоты осуществляется благодаря нелинейности и диссипативности и имеет место при условии, что частоты и0 и и не слишком отличаются друг от друга, то есть для некоторого ограниченного диапазона частотных расстроек, который называется полосой захвата. Полоса захвата зависит от свойств автогенератора и от амплитуды внешней силы. Захват частоты может наблюдаться в автоколебательных системах любой физической природы и при различных периодических внешних воздействиях. Впервые же оно было обнаружено и объяснено для томпсоновского генератора с синусоидальным воздействием. Другой распространённый термин для захвата частоты синхронизация автогенератора внешней силой.

Наиболее полно развита теория синхронизации колебаний для квазигармонических колебаний в слабо нелинейных системах [1-3]. Целью настоящей работы является изучение явления синхронизации (захвата) лампового генератора внешней гармонической силой, частота и которой близка к собственной частоте генератора $\omega_0$.


\section{Структурная схема и обобщенная модель генератора}
Схема генератора и графики анодно-сеточной и сеточной характеристик лампы приведены на рис.1. Уравнения генератора, составленные без учета реакции анода и межэлектродных емкостей, имеют вид
\begin{equation}
  \label{eq1}
  1
\end{equation}
где $i_a$ и $i_c$ -- анодный и сеточный токи лампы, зависящие в общем случае от анодного и сеточного напряжения. Если напряжения на сетке положительны, но нс велики, то сила тока в цепи сетки будет мала по сравнению с силой анодного тока $i_c \ll i_a$ (при отрицательных напряжениях на сетке ток в цепи сетки практически исчезает). 

На практике к условию отсутствия сеточного тока можно подойти достаточно близко, выбирая режим работы лампы так, чтобы напряжение на сетке нс переходило в область положительных значений. В этом случае наличием тока в цепи сетки можно пренебречь $i_c(u) = 0$. Далее будем рассматривать именно такой случай.

Рис. 1. Схема лампового генератора с гармоническим воздействием (а), графики анодносеточной и сеточной характеристик лампы (б)

Для математического анализа уравнения (1) необходимо иметь характеристики электронной лампы в явном виде. Наиболее обычный путь - представление функции $i_a(u)$  при помощи полинома. Во многих случаях характеристики лампы могут быть с достаточной точностью аппроксимированы полиномами третьей или пятой степени, симметричными относительно рабочей точки (см. [4], стр.540.), при этом переменное напряжение на сетке лампы удобно рассматривать относительно постоянного сеточного смещения $E_0$, т.е.  $u = u +E_0$. Напомним, что при аппроксимации характеристики лампы полиномом третей степени возникающий автоколебательный режим всегда является мягким, а в случае полинома пятой степени он может быть жёстким (см. [5], стр.684).

\section{Синхронизация мягкого режима}
\subsection{Модели генератора в мягком режиме возбуждения}
Рассмотрим случай, когда автономный генератор находится в режиме мягкого самовозбуждения. Для этого случая достаточно характеристику лампы аппроксимировать полиномом 3-й степени $i_a(u) = i_0 + S_0u - \gamma u^3$, где напряжения отсечки и насыщения расположены симметрично относительно рабочей точки. Для этих напряжений крутизна характеристики лампы обращается в ноль, в частности,
\begin{gather}
  \label{eq2}
  1
\end{gather}
Из (2) следует, что $\gamma=\frac{S_0}{3u_0^2}$, а зависимость крутизны характеристики лампы приобретает вид $S(u) = S_0(1 — u^2/u_0^2)$.
Для исследования модели (1) воспользуемся методом малого параметра (методом Ван-дер-Поля), который справедлив для систем, близких к линейному осциллятору [6]. Чтобы условие близости выполнялось, необходимо позаботиться о соответствующем подборе параметров. Введем безразмерное время $\tau=\ometa t$ и перепишем систему уравнений (1) в виде одного уравнения:
\begin{gather}
  \label{eq3}
  1
\end{gather}
Введем безразмерные параметры
\begin{equation}
  1
\end{equation}
и переменную
\begin{equation}
  1
\end{equation}
С учетом принятых обозначений уравнение (3) преобразуется к виду
\begin{gather}
  \label{eq4}
  1
\end{gather}
Для применимости метода Ван-дер-Поля необходимо, чтобы параметры $\mu$ и $\varepsilon_0$ были малыми ($\mu, \varepsilon_0 \ll 1$). Малость параметра $\mu$ обеспечивается рассмотрением генератора вблизи границы самовозбуждения (генератор слабо возбужден), a $\varepsilon_0$ - малостью амплитуды внешней ЭДС. Далее введем в рассмотрение безразмерные параметры $\xi$ и $\varepsilon$: $\mu\xi=(\omega^2-\omega_0^2)/\omega^2$ -- относительная расстройка частот и $\mu \varepsilon = \varepsilon_0$ -- амплитуда внешнего воздействия. Новые параметры £ и £ не малы (порядка единицы), малыми являются произведения д£ и д£. С учетом введенных параметров уравнение (4) принимает вид
\begin{gather}
  \label{eq5}
  5
\end{gather}
Нелинейная динамическая система (5) определена в неавтономном фазовом пространстве U = {x,X,т(mod2n)} и трехмерном пространстве параметров Л={0 < д ^ 1, £ > 0, —то < £ < +то}.
Далее, по методу Ван-дер-Поля, от (5) переходим к автономной системе укороченных уравнений1:
\begin{gather}
  \label{eq6}
  6
\end{gather}
которая определена на фазовом полуцилиндре V+={ф(тос12я), p>0} и зависит от двух параметров £ > 0 и —то < £ < то. Параметр д исключен из рассмотрения заменой т1 = дт.

\end{document}
\subsection{Динамические режимы генератора и аттракторы моделей}
При изучении динамики автотогенератора, находящегося под внешним гармоническим воздействием, в поле зрения исследователя находятся два режима: режим синхронизации и режим биений2. Эти режимы в фазовых пространствах динамических моделей имеют свои образы, которые для моделей (5) и (6) различны.
Режим синхронизации в неавтономном фазовом пространстве U модели (5) представляется устойчивой периодической траекторией с периодом TT = 2пm7 где отношение ££ - рациональное число. Если ££ = 1, то говорят
1 Формально переход от (5) к (6) осуществляется заменой x = р cos п X = —р sin п П = т + ф, где р > 0, по формулам
1 Г2п
P (р,ф) = R(p cos п, —р sin — ^)sin ndn,
2п Jо
1 г2п
Q(fj,p) = — -—  R^ cos п, —рsin — ф) cos ndn,
2пр Jо
Биения - периодические изменения во времени амплитуды колебания, возникающего при сложении двух гармонических колебаний с близкими частотами. Биения появляются вследствие того, что величина разности фаз между двумя колебаниями с различными частотами всё время изменяется так, что оба колебания оказываются в какой-то момент времени в фазе, через некоторое время в противофазе, затем снова в фазе и т.д. Соответственно амплитуда результирующего колебания периодически достигает то максимума, равного сумме амплитуд складываемых колебаний, то минимума, равного разности этих амплитуд.
7
о синхронизации на основном тоне (траектория S\ на рис. 2г), в противном случае синхронизация осуществляется на гармониках или субгармониках. В автономном фазовом пространстве V + модели (6) образом режима синхронизации являются устойчивое состояние равновесия (точка O\ на рис. 2а). В модели (6) устойчивое состояние равновесия соответствует режиму синхронизации на основном тоне. Далее, говоря о синхронном режиме, мы будем иметь виду синхронизацию на основном тоне.
Образом режима биений в неавтономном фазовом пространстве модели (5) является устойчивый инвариантный тор (рис. 2д). Проекция инвариантного тора модели (5) на плоскость (x, X) и соответствующая этому тору осциллограмма (x, г) приведены на рис. 2е. Визуальный анализ осциллограммы указывает на явное наличие в рассматриваемом колебании амплитудной модуляции, кроме того это колебание также модулировано по частоте, поскольку точки пересечения им нулевого уровня располагаются неэквивалентно.
В автономном фазовом пространстве V + модели (6) режиму биений соответствуют устойчивые предельные циклы. В силу цилиндричности фазового пространства модели (6) предельные циклы в этом пространстве могут быть двух типов [7]: циклы 1-го рода (колебательные), не охватывающие фазовый цилиндр V + {L\ на рис. 26), циклы 2-го рода (вращательные), охватывающие фазовый цилиндр V + (L2 на рис. 2в). Размер и период предельного цикла характеризуют закон модуляции - амплитуда предельного цикла определяет амплитуду модуляции., а величина Qm = 1/Tc определяет частоту модуляции, где Tc - период предельного цикла (см. рис. 2е).
Далее обсудим, как тип предельного цикла отражается на режиме биений. Колебания 1-го и 2-го рода различаются поведением переменной ф, в первом случае она колеблется около некоторого среднего значения, во втором неограничено возрастает. Переменная ф определяет разность фаз колебаний автогенератора и внешнего воздействия, а её производная ф - разность частот. Усредненная на периоде Tc перемениая ф в случае цикла 1-го рода равна нулю (<ф>=0), в случае цикла 2-го рода она не равиа нулю (<ф >= 0). Отсюда следует что, в случае цикла 1-го рода усредненная частота (средняя частота) колебаний генератора совпадает с частотой внешнего сигнала, т.е. частотная модуляция в режиме биений в «среднем» отсутствует. Режим биений, в котором средняя частота колебаний генератора совпадает с частотой внешнего сигнала, принято называть режимом фазовой синхронизации [8]. В системах автоматической фазовой синхронизации такой режим называют квазисинхронным режимом [9]. В случае предельного цикла 2-го рода средняя частота колебаний генератора и частота внешнего сигнала не совпадают - синхронизации нет (асинхронный режим).
8
(г)
(д)
2
X
-2
-2  X 2
(е)
Рис. 2. Примеры фазовых портретов автономной модели (a-в); неавтономной модели (г); качественное изображение инвариантного тора в невтономном фазовом пространстве (д); (х, у)-проекция инвариантного тора модели (5) и соответствующая ему осциллограмма (е)
9
\subsection{Динамика укороченной модели}
2.3.1.  Локальная устойчивость синхронного режима
Рассмотрим состояния равновесия модели (6), координаты которых есть решения системы уравнений
Р(1 - Р2) = — sin ф,  (7)
£р =  £ COS ф.
Возводя оба уравнения (7) в квадрат и складывая, получим уравнение для «резонансных кривых»3
р2(1 - р2)2+ер2 = £2, (8)
из которого можно получить зависимости квадрата амплитуды колебаний р2 от расстройки £ и параметра £. Рассмотрим вид и устойчивость резонансных кривых, определяемых уравнением (8). Полагая Z1(р2) = р2(1 — р2)2 и Z2(p2,£2,£2) = £2 — £2р2, строим графики этих функций (рис. 3). Пересечения графиков Z1 и Z2 определяют координаты р* состояний равновесия Oi модели (6) и точки на резонансных кривых. Варьируя £ и £, получаем серию резонансных кривых симметричных относительно прямой £ = 0 (рис. 4а). Проанализируем поведение этих кривых при различных значениях £.
Рис. 3. Вид вспомогательных кривых Z^2) и Z2(р2,е2, £2) для построения резонансных кривых в случае мягкого автоколебательного режима
3Термин «резонансная кривая» взят в кавычки и носит чисто условный смысл, ибо понятие резонанса для автоколебательных систем не определено [6]. Резонансные кривые характеризуют зависимость амплитуды периодических колебаний от частоты внешнего воздействия, поэтому эти кривые так же называют амплитудно-частотными характеристиками неавтономного генератора - АЧХ [10]. В настоящей работе термины резонансная кривая и АЧХ будут употребляться как равноправные.
10
• е1 < 4/27.  При е1 = 0 (внешнее воздействие отсутствует) резонансная кривая вырождается в точку (£=0, р2=1) и прямую р1 = 0. При 0 < е1 < 4/27 резонансная кривая состоит из двух раздельных кривых: одна есть замкнутая кривая, охватывающая точку (^=0,р2=1), другая располагается вблизи оси абсцисс (на рис. 4а линия 1). Здесь в зависимости от £ будет либо три (на рис. 4а светло-серая область), либо одно равновесное состояние.
• е1 = 4/27. Пр и £ = 0 замкнутая кривая и линия 1 касаются друг друга в точке (£=0, р2=1/3), образуя единую петлеобразную кривую (линия 2). Здесь при £ = 0 система (6) имеет два состояния равновесия, при £ = 0 -либо три (на рис. 4а серая и светло-серая области), либо одно равновесное состояние.
• 4/27 < е1 < 8/27. Резонансная кривая состоит из одной линии 3. При
£ = 0 модель (6) имеет только одно состояние равновесия. При £=0 область с одним равновесным состоянием разрывается областями с тремя состояниями равновесия (на рис. 4а две области со штриховкой).
• е1 = 8/27. Резонансная кривая есть линия 4. Области со штриховкой
вырождаются в линии £ = здесь модель (6) имеет два состояния
равновесия, при остальных £ - одно.
• е1 > 8/27.  Резонансная кривая однозначно определена для любых £ (линия 5), модель (6) имеет только одно состояние равновесия.
Каждой точке резонансной кривой отвечает состояние равновесия укороченной системы (6), а при малых $\mu$ периодическое колебание исходной модели (5). Режим периодических колебаний устойчив, если устойчиво соответствующее состояние равновесия укороченной системы. Устойчивость состояний равновесия определим по корням характеристического уравнения.
Пусть р*,ф* - координаты состояния равновесия модели (6), тогда характеристическое уравнение имеет вид
где
det
( др (рф) др
dQ(P,p) \ др
X
др (Р,Ф) дф
dQ(P,p)
дф
X
(р* *,ф*)
0,
дР (р*,ф*) др
дР (р,,Ф*) дф
1 - З^)1,
е cos ф = с учетом (7)  = £р*,
11
(б)
Рис. 4. Резонансные кривые (а) и области устойчивости резонансных кривых (б) в случае мягкого автоколебательного режима
дЯ(р*,Ф*) £ cos ф £
= = = , др  (р*)2 Р
(9)
dQ(p *,ф *)
£ Sin ф*
= с учетом (7)  =1 — (р*)2.
дф  (р*)
Перепитом характеристическое уравнение в виде квадратного уравнения
Л2 +  2 [2(р*)2 — 1] Л  + [1  — (р*)2]  [1 — 3(р*)2] + £2 = 0.  (Ю)
Необходимое и достаточное условие устойчивости состояний равновесия -положительность коэффициентов характеристического уравнения (10), т.с. область, расположенная выше линии а = (р*)2 — 1/2 = 0 и вне эллипса А = 1 — (р*)2 1 — 3(р*)2 + £2 = 0. Разбиение плоскости (£,р) на области
12
с различными типами состояний равновесия представлено на рис. 46. Область существования устойчивых состояний равновесия (синхронного режима) ограничивают кривые: а = 0 вне области седел - бифуркации Андронова-Хопфа и А = 0 - бифуркации двукратного состояния равновесия. Диапазон расстроек частот £, где существует синхронный режим, называется полосой синхронизации. В зависимости от значений амплитуды £ сигналы делятся на слабые и сильные. Сигнал называется слабым, если соответствующая ему резонансная кривая пересекает прямую а = 0, сильным - кривую А = 0. Деление сигналов на сильные и слабые обусловлено различными сценариями возникновения режима биений, эти сценарии будут рассмотрены ниже.
2.3.2.  Структура плоскости параметров (£, Д
На рис. 5 приведено разбиение плоскости параметров (ДД модели (6) на области с различными фазовыми портретами. Этот рисунок качественно отражает результаты компьютерного моделирования, проведенного с помощью программного комплекса ДНС [11]. Результаты качественного исследования структуры пространства параметров модели (6) можно найти в [2,12]. Поскольку разбиение симметрично относительно оси ординат, то на рис. 5 изображена лишь правая часть разбиения (область £>0). Представленная структура образована кривыми следующих бифуркаций:
• штрихпунктирная линия соответствует бифуркации двукратного состояния равновесия. На этой линии выполняется условие А(£, Д = 0, т.е. хотя бы один из корней характеристического уравнения (10) обращается в ноль (Ах • A2=0,Im(А1;2)=0). Эта линия ограничивает область существований трех состояний равновесия: 0х,02 и 03 (рис. 6а). На верхней границе этой области сливаются состояния равновесия 02 и 03, на нижней - 02 и 0х;
• линия 1 - бифуркации Андронова-Хопфа, смена устойчивости состояния равновесия 0х (03). На этой линии выполняется условие а(Д Д = 0, т.е. корни характеристического уравнения (10) располагаются на мнимой оси (Re(A1;2)=0,Im(А1;2)=0). Состояние равновесия теряет устойчивость при пересечении линии 1 сверху вниз (слева направо). Поскольку первая ляпуновская величина на бифуркационной кривой отрицательна, то потеря устойчивости сопровождается мягким рождением устойчивого колебательного предельного цикла Lx (см. рис. 6в,г); •
• линия 2 - бифуркации петли сепаратрис седла 02. Пересечение этой кривой с ростом £ или £ приводит к рождению колебательного цикла Lx,
13
Рис. 5. Структура плоскости параметров (фе) модели (6): (а) - качественное изображение, (б) фрагмент реальной картины
(а)
(в)
(д)
(б)
(г)
(е)
Рис. 6. Грубые фазовые портреты модели (6) и их аналоги в декартовой системе координат (x = р cos ф,у = р sin ф)
14
так как седловая величина а = А1 + А2 < 0, то возникающий цикл устойчив. Концевыми точками для бифуркационной кривой служат точки а и b. В точке а выполняются условия теоремы Богданова [13] - состояние равновесия О2 имеет два нулевых корня (A1=A2=0), в этой точке собираются вместе линия 1 и линия 2, на штрихпунктирной линии слева от точки а(0.5, 0.5) двукратное состояние равновесия имеет устойчивую узловую часть, справа - неустойчивую. В точке b(0.5507,0.5237) петля сепаратрис седла превращается в петлю сепаратрис седло-узла;
• линия 3 есть участок штрихпунктирной линии, расположенный между точками b(0.5507,0.5237) и с(0.5381278,0.5132685). Эта линия соответствует бифуркации петли сепаратрис седло-узла 1-го рода (петля не охватывает цилиндр V+), в результате которой появляется устойчивый колебательный предельный цикл Li;
• линия 4 - участок штрихпунктирной линии, расположенный ниже точки с, соответствует бифуркации петли сепаратрис седло-узла 2-го рода (петля охватывает цилиндр V+). Из этой петли рождается устойчивый вращательный предельный цикл L2;
• линия 5 разделяет области существования колебательного L1 и вращательного L2 предельных циклов. При значениях параметров на этой линии совпадают инвариантные кривые Is и Iu (см. рис. 6д,е).
Бифуркационные кривые делят плоскость (ф ф на шесть областей D1 — —D6. Соответствующие этим областям фазовые портреты модели (6) приведены на рис. 6. В области D1 структуру фазового пространства определяют: устойчивый узел (фокус) Оц неустойчивый узел (фокус) Оз и седло O2 (рис. 6а), а также инвариантные кривые Iu и 1^, которые при £ = 0 вырождаются в прямые. В декартовой системе координат особые точки с координатами (ф = п/2,р = 0) и (ф = 3п/2,р = 0) преобразуются в проходимую точку О с координатами (х = 0, у = 0). При значениях параметров из области D2 все фазовые траектории модели (6) стремятся к устойчивому состоянию равновесия О1 (рис. 66). В области D3 система (6) имеет два устойчивых состояния равновесия О1 и О3 с бассейнами притяжения П(Оф и П(О3) соответственно (рис. 6в). Границами П(Оф и П(О3) служат входящие сепаратрисы V{ и V2s седл а О2. При значениях из о бласти D4 в фазовом пространстве модели (6) имеют место два аттрактора (рис. 6г): состояние равновесия О3, соответствующее режиму синхронизации, и колебательный предельный цикл L1, характеризующий режим биений. Таким образом, в областях параметров D3 и D4 система (6) демонстрирует бистабильное поведение. На рис. 5а области с бистабильным поведением отмечены серым цветом, в этих областях
15
состояние генератора определяется не только значениями параметров, но и начальными условиями. В областях параметров D5 и D6 фазовый портрет модели (6) характеризуется четырьмя особыми траекториями: предельным циклом (колебательным L1 в области D5 и вращательным L2 в области D6), неустойчивым состоянием равновесия O3 инвариантными кривыми IM и Is .
2.3.3.  Сценарии нарушения синхронизации
Синхронный режим при увеличении £ всегда сменяется режимом биений. В режиме биений колебания являются модулированными и характеризуются амплитудой и частотой модуляции. Сценарии возникновения модулированных колебаний зависят от амплитуды внешнего сигнала е. Как правило, выделяют два механизма нарушения синхронизации и возникновения модулированных колебаний, различаемые по амплитуде внешнего сигнала. Если е2 < 1/4 входной сигнал считают слабым, а при е2 > 8/27 - сильным. Эти случаи характеризуются отсутствием гистерезисных явлений при вариациях параметра £, но различаются сценариями возникновения модуляции. Существует небольшой интервал 1/4 < е2 < 8/27, где поведение генератора при вариациях £ не однозначно. Далее остановимся на рассмотрении того, что происходит с синхронными колебаниями при увеличении £ при различных значениях параметра е.
А. Случай слабого сигнала, е2 < 1/4. В случае слабого сигнала исчезновение синхронного режима и возникновение режима биений с точки зрения теории бифуркаций связано с бифуркацией петли сепаратрис седло-узла (рис. 7а). Фазовые картины укороченной системы уравнений (6) в окрестности и на границе полосы синхронизации приведены на рис. 6. При расстройках £ < |£i| единственным аттрактором системы (6) является состояние равновесия O1 (рис. 6а), оно и определяет режим синхронизации. По мере увеличения расстройки £ состояние равновесия O1 сближается с седловым состоянием равновесия O2, при £ = £2 они сливаются, образуя негрубое состояние равновесия O1-2 седло-узел. При этом выходящая сепаратриса седло-узла возвращается к O1-2, касаясь ведущего направления, т.е. в фазовом пространстве V + имеет место петля сепаратрис седло-узла. При дальнейшем увеличении £ состояние равновесия O1-2 исчезает (исчезает режим синхронизации), а в фазовом пространстве модели (6) появляется устойчивый предельный цикл (рис. 6д,е) - возникает режим биений. Цикл, рожденный из петли сепаратрис седло-узла, обладает следующими свойствами: в окрестности бифуркационного значения параметра £ = £2 он повторяет форму петли, т.е. амплитуда этого предельного цикла в общем случае не мала; период Tc предельного цикла при £ ^ £2 стремится к бесконечности. Эти свойства определяют законы
16
Рис. 7. Резонансная кривая (а), зависимости амплитуды (б) и частоты (в) модуляции от расстройки £, вид осциллограмм на границе полосы синхронизации: синхронный режим (г) и режим биений (д) в случае слабого сигнала
возникновения модуляции при выходе из полосы синхронизации: амплитуда модуляции изменяется скачком (рис. 76), частота модуляции шт = 1/Tc растет от нуля (рис. 7в). При движении в обратную сторону, от режима биений к синхронному режиму с уменьшением £, наблюдается обратная картина: при £ = £2 предельный цикл исчезает в петлю сепаратрис седло-узла - режим биений исчезает, появляется глобально устойчивое состояние равновесия Оц
17
определяющее режим синхронизации.
Б. Случай сильного сигнала, е1 > 8/27. В случае сильного сигнала исчезновение синхронного режима и возникновение режима биений сопровождается бифуркацией Андронова-Хопфа. Фазовые картины укороченной системы уравнений (6) в окрестности и на границе полосы синхронизации приведены на рис. 6. При расстройках £ < |£2| состояние равновесия О1 устойчиво. При увеличении £ и переходе через границу £ = £2 состояние равновесия Oi теряет устойчивость (синхронизация нарушается), а в фазовом пространстве модели (6) рождается устойчивый предельный цикл (рис. бд), характеризующий режим биений. Поскольку первая ляпуновская величина в момент бифуркации отрицательна, то рождение цикла происходит мягко, т.е. амплитуда цикла (амплитуда модуляции) растет от нуля (рис. 86). Что касается периода Tc предельного цикла, то в момент рождения он конечен, т.е. частота модуляции при выходе из полосы синхронизации изменяется скачком (рис. 8в). При движении в обратную сторону наблюдается обратная картина: при £ = £i устойчивый предельный цикл стягивается в точку Оц в результате чего состояние равновесия О1 приобретает устойчивость - режим биений трансформируется в синхронный.
В. Диапазон 1/4 < е1 < 8/27 характеризуется неоднозначным поведением генератора на границе полосы синхронизации £ = £2 при вариациях £. Пусть начальное значение £ принадлежит области Dy где колебания генератора синхронизированы внешним сигналом. При увеличении £ система (6) переходит из области D1 в область D4 и далее в область D5, где реализуется режим биений. Переход из D1 (рис. 6а) в область D4 (рис. 6г) сопровождается рождением устойчивого предельного цикла (£ = ф), отвечающего за режим биений. Появление предельного цикла L1 не нарушает синхронизации, так как для реализации режима биений в области D4 необходимо, чтобы начальные условия модели (6) принадлежали области притяжения n(L1), при переходе же из D1 в D4 состояние системы (6) остается в точке О1. При дальнейшем увеличении £ состояние равновесия О1 приближается к О2, на границе областей D4 и D5 они сливаются (при £ = ф) и исчезают. В результате фазовые траектории из окрестности О1 устремляются к предельному циклу L1, синхронизация нарушается, устанавливается режим биений. Так как в момент нарушения синхронизации цикл L1 имеет не малую амплитуду и конечный период, то амплитуда и частота модуляции при выходе из полосы синхронизации изменяются скачком (рис. 9).
Теперь рассмотрим поведение системы при обратном ходе: из области D5 в область D1 через D4. При уменыиении £ на границе областей D5 и D4 (при £ = ф) появляется состояние равновесия О1. Однако это не приводит
18
(а)
(б) (в)
О Т 1500
(г)
(д)
Рис. 8. Резонансная кривая (а), зависимости амплитуды (б) и частоты (в) модуляции от расстройки £, вид осциллограмм на границе полосы синхронизации: синхронный режим (г) и режим биений (д) в случае сильного сигнала
19
Рис. 9. Гистерезис на границе полосы синхронизации
к восстановлению режима синхронизации, поскольку при обратном ходе в области D4 состояния модели (6) определены на цикле L1. Режим биений будет сохраняться до исчезновения цикла Ь\. При £ = £1 цикл Ь\ влипает в петлю сепаратрис седла О2 и при дальнейшем уменьшении £ исчезает. В результате фазовые траектории из окрестности L1 устремляются к состоянию равновесия Оц режим биений разрушается, режим синхронизации восстанавливается. Происходит захват в режим синхронизации. Отмстим, что на границе области захвата £ = ф амплитуда ци кла L1 определяется размером петли сепаратрис (рис. 96), а частота цикла равна нулю (рис. 96).
Таким образом, из рассмотренного случая следует, что в модели (6) существуют такие значения £, при которых нарушение и восстановление режима синхронизации происходит при разных значениях расстройки £. Полосу частот (—£2, £2), где режим синхронизации существует, называют полосой удержания синхронного режима. Полосу частот — £i,£i), где режим синхронизации наступает при любых начальных условиях, называют полосой захвата в синхронный режим. Если полоса захвата меньше полосы удержания, то существуют интервалы (—£2, — ф) и (£ь£2), где модель (6) демонстрирует бистабильное поведение, которое обуславливает гистерезисные явления при вариациях £. В случае слабого и сильного сигналов полосы удержания и захвата совпадают, здесь при вариациях £ нарушение и восстановление режима синхронизации происходит при одном и том же значении £1 = £2.
2.4. Динамика исходной модели
Рассмотрим динамику неавтономной модели (5). Заметим, что анализ движений этой модели (5) является более сложной задачей, чем исследование фазовых траекторий модели (6) |14|. Это обусловлено тем, что модель (5) относится к классу многомерных динамических систем, которые допускают существование квазипсриодичсских и хаотических колебаний; спектр возможных бифуркаций у многомерных моделей существенно шире, чем у двумерных;
20
наконец, некоторые бифуркации многомерных систем до сих пор не имеют строгих алгоритмов идентификации. Несмотря на вышесказанное, при малых $\mu$ между особыми траекториями моделей (6) и (5) существует определенная связь, которая отражена в таб. 1.
Таблица 1.
модель (6)  модель (5)  сечение Пуанкаре
состояния равновесия 0\, О2, Os периодические траектории Si,S2,Ss неподвижные точки Si,S2,Ss
сепаратрисы Vu Vu Vs Vs двумерные многообразия WW?, Wf, W2s одномерные кривые WWWf, W2s
периодические траектории Li (L2)  инвариантный тор T  замкнутая инвариантная кривая T
Для иллюстрации динамики многомерных систем часто используют картины отображения Пуанкаре, которые получаются в результате пересечения фазовых траекторий динамической системы с некоторой секущей. Мы также будем использовать эти картины, поскольку, во-первых, в неавтономном фазовом пространстве модели (5) существует секущая т0 = тоd(2n), которую все фазовые траектории модели (5) пересекают трансверсально, т.е. у модели (5) всегда существует двумерное отображение Пуанкаре {х(то), х(т0)}. Во-вторых, при малых $\mu$ между особыми траекториями модели (6), представленными в декартовой системе координат, и особыми траекториями двумерного отображения Пуанкаре, порождаемого траекториями модели (5), существует определенная связь (см. таб. 1). В частности, образами состояний равновесия, сепаратрис и предельных циклов модели (6) на секущей Пуанкаре являются соответственно неподвижные точки, одномерные инвариантные многообразия и замкнутые инвариантные кривые. Тип периодической траектории модели (5) (тип неподвижной точки отображения Пуанкаре) совпадает с типом состояния равновесия модели (6), инвариантные многообразия и торы имеют ту же устойчивость, что сепаратрисы и предельные циклы модели (6).
На рис. 2г приведен один из фазовых портретов модели (5). Он содержит три периодических движения с периодом TT=2п {Tt=2л/и): Si - устойчивый узел или фокус, S3 - неустойчивый узел ил и фокус, S2 - седло. Неустойчивые инвариантные многообразия Wi и W2U седлa S2 замыкаются на устойчивом периодическом движении Si, устойчивое инвариантное многообразие W2 при т ^ —ж стремите я к S3, a W( уходит в бесконечность.
Портрет отображения Пуанкаре имеет три неподвижные точки. Их тип и обозначения совпадают с периодическими траекториями, их порождающими. Фазовый портрет модели (6) в декартовой системе координат, являющийся аналогом рассматриваемого сечения Пуанкаре, представлен на рис. ба. Как
21
отмечалось выше, аналогом состояний равновесия 0i,02,03 модели (6) являются неподвижные точки Si, S2, S3 на секущей Е. Аналогом сепаратрис седла 02 служат сепаратрисы седловой неподвижной точки S2. На практике эти сепаратрисы представляют собой следы пересечения инвариантных многообразий WU W2, W( и W2 седлового периодического решения S2 секущей плоскостью Е.
Таким образом, изучение динамики неавтономной модели (5) при малых д может быть сведено к поиску неподвижных точек отображения Пуанкаре, построению инвариантных многообразий, анализу бифуркаций особых траекторий и построению на плоскости — е) бифуркационных кривых, аналогичных линиям 1-5 на рис. 5. Результаты компьютерного анализа модели (5) приведены на рис. 10. Здесь на плоскости — е) выделены области D\ — D6 с различным динамическим поведением. Картины отображения Пуанкаре для выделенных областей представлены на рис. 11.
В области Di система (5) имеет три периодических решения: устойчивое Si, седловое S2 и неустойчивое S3 (рис. 2г). Соответствующая этой области параметров картина отображения Пуанкаре приведена на рис. 11а, она содержит три неподвижные точки: устойчивую Si, седловую S2 и неустойчивую S3. При выходе из об ласти D1 через штрихпунктирную линию седловое периодическое решение S2 сливается с Si или S3 и исчезает. В момент бифуркации один из мультипликаторов периодического решения (неподвижной точки) становится равным +1, штрихпунктирная линия соответствует касательной (седло-узловой) бифуркации.
В области D2 единственным аттрактором модели (5) является периодическое решение Si. Картина отображения Пуанкаре для этой области представлена на рис. 116, она содержит одну неподвижную точку Si и «следы» семи траекторий модели (5) с начальными (т=0) условиями в точках l(x= —1.0, x= — 0.25) 2—1.0, -0.4— 3—1.0, -0.6— 4—1.0, —0.85); 5—1.0, -0.9— б(—1.5, -0.9— 7(—2.0, —0.90). Фазовые траектории 1-7 асимптотически приближаются к периодическому решению Si, а их «следы» (траектории точного отображения) - к неподвижной точке Si.
При значениях параметров из области D3 в фазовом пространстве модели
(5) существуют устойчивые периодические траектории Si и S3, обе эти траектории определяют режим синхронизации. Из картины отображения Пуанкаре на рис. Ив видно, что при равномерном распределении начальных условий вероятность наступления синхронного режима S3 меньше, чем Si, так как бассейн притяжения П(5'з) меньше, чем n(Si). Границами бассейнов притяжения n(Si) и n(S3) являются устойчивые инвариантные многообразия WS и WS седлa S2.
22
Рис. 10. Структура плоскости (£,е) модели (5) при $\mu$ = 0.1
Рис. 11. Картины отображения Пуанкаре модели (5) при значениях параметров: ^=0.1, £=0.532, е=1.031  £=0.53,е=1.041  £=0.53,е=1.034 (в), £=0.535,е = 1.036
£=0.53,е=1.023 (д), £=0.53,е=1.0
23
В области D4 система (5) также демонстрирует бистабильное поведение, однако здесь, в отличие от области D3, возможен как режим синхронизации, так и режим биений. Картина отображения Пуанкаре для области D4 представлена на рис. 11г. Она содержит устойчивую неподвижную точку Si, устойчивую инвариантную кривую T, седловую точку S2 с инвариантными многообразиями, а также неустойчивую точку S3. Вероятность наступления синхронного режима S3 зависит от поведения инвариантных многообразий Wi и  седлa S2, которые определяют бассейны притяжения n(Si) и П(Т).
На рис. 11 д и рис. Не приведены картины отображения Пуанкаре для областей D5 и D6 соответственно. Качественно эти картины не отличаются друг от друга, обе картины содержат замкнутую устойчивую инвариантную кривую T, внутри которой располагается неустойчивая неподвижная точка S3. Инвариантная кривая T свидетельствует, что в фазовом пространстве модели (5) существует глобально устойчивый инвариантный тор, определяющий режим биений. Картины на рис. 11 д и рис. Не построены при близких значениях параметров модели, можно отметить существенные различия в размерах инвариантной кривой (инвариантного тора). Однако основным критерием разделения пространства параметров модели (5) на области D5 и D6, является совпадение средней частоты генерируемых колебаний йт с частотой внешнего сигнала. Практически От можно вычислить по формуле
йт
N
WW)
1
где т - периоды генерируемых колебаний (в ре жиме биений T=Tj для любых i = j), N - большое число. На рис. 12 приведены зависимости AT=TCp—2п
- сдвига среднего периода генерируемых колебаний относительно периода внешнего сигнала (N = 2000). Здесь цифрами 1-6 отмечены бифуркационные значения £, где возникает режим биений при е=1.; 1.1; 1.2; 1.3; 1.4; 1.5 соответственно. Точки 2-6 соответствуют бифуркации Неймарка-Сакера, точка 1
- касательной бифуркации.
На линии 1 комплексно-сопряженные мультипликаторы периодического решения S1 выходят на единичную окружность (модуль мультипликаторов обращается в единицу), т.е. периодическое решениеS1 проходит через бифуркацию Неймарка-Сакера. В результате этой бифуркации устойчивое периодическое решение теряет устойчивость, при этом в его окрестности возникает устойчивый инвариантный тор T. На линии 2 тор T влипает в гомоклиническую петлю, образованную инвариантными многообразиями седлового решения S2.
24
0.40 1 AT
Рис. 12. Отклонения TCp среднего периода генерируемых колебаний от периода внешнего сигнала, рассчитанные по траекториям модели (5) при д = 0.1
\section{Синхронизация жесткого режима}
Рассмотрим случай, когда автономный Д=0) генератор находится в режиме жесткого самовозбуждения. Фазовый портрет, характеризующий этот режим, представлен на рис. 13а. На портрете устойчивое состояние равновесия О и устойчивый предельный цикл L разделены неустойчивым циклом S. Траектории, начинающиеся внутри неустойчивого предельного цикла, будут идти к состоянию равновесия, а траектории, начинающиеся вне цикла S, будут наматываться на устойчивый предельный цикл L. В генераторе в зависимости от начальных условий будет устанавливаться или состояние равновесия или автоколебания. Для возникновения автоколебаний в автогенераторе системе необходимо дать некоторый «толчок», выводящий начальные условия из области притяжения состояния равновесия О (на рис. 13а из серой области). Существование жесткого режима возможно в генераторе, если характеристика лампы аппроксимируется полиномом нс ниже пятой степени. Пусть $i_a$(u) = i_0 + S_0u + yU — (5иъ.
В жестком режиме поведение неавтономной системы, также как и автономной, зависит от начальных условий. Если генератор находится в автоколебательном режиме, то, при наличии внешнего периодического воздействия, возможно захватывание частоты и реализация синхронного режима. Если же генератор нс возбужден, то система ведет себя как нелинейный колебательный контур, в котором при изменении частоты внешней силы можно наблюдать явление нелинейного резонанса.
25
2
tP
Ч. Ч» 0 ^2
(B)
Рис. 13. Фазовый портрет (а), вид вспомогательной кривой для построения резонансных кривых (б), резонансные кривые и области их устойчивости (в) в случае жесткого автоколебательного режима
26
\subsection{Динамика укороченной модели}
Методика исследования явления синхронизации в случае жесткого режима аналогична случаю мягкого режима [15]. Здесь мы ограничимся показом резонансных кривых и анализом структуры плоскости (£,£).
3.1.1.  Резонансные кривые
На рис. 136 качественно изображен вид функции Zi(p2) и Z2(p2) для определения координат состояний равновесия укороченной системы уравнений в случае аппроксимации характеристики лампы полиномом пятой степени. Из рисунка видно, что укороченная система уравнений в зависимости от параметров £ и £ может иметь от одного до пяти состояний равновесия. Анализ координат и устойчивости этих состояний равновесия дает картину резонансных кривых, изображенную на рис. 13в. Ветви АЧХ пронумерованы в порядке увеличения £, чем больше значение параметра £, тем больше номер кривой. При малых £ АЧХ состоит из трех кривых (линии 1), две из которых образуют замкнутые контура. При увеличении £ замкнутые контура соприкасаются, образуя контур в виде восьмерки (линия 2). Далее восьмерка превращается в замкнутый контур, который при увеличении £ соприкасается с линией вблизи оси абсцисс, образуя единую петлеобразную кривую (линия 3). При больших £ резонансные кривые представляются едиными линиями, которые сначала имеют интервалы по £ с неоднозначным поведением (линия 4), потом эти интервалы исчезают (линия 5). В случае неоднозначного определения резонансные кривые всегда имеют точки пересечения с границей существования седел А(£, £) = 0 (пунктирные и штрихпунктирные линии). Резонансные кривые, однозначные для любых £, пересекают только границы а(£,£) = 0, разделяющие устойчивые и неустойчивые фокусы.
Из рис. 136 видно, что при фиксированной амплитуде сигнала £ могут быть устойчивы две ветви АЧХ (верхняя и нижняя). При больших начальных условиях имеет место режим синхронизации, при малых амплитудах происходят вынужденные колебания. Анализ первой ляпуновской величины показывает, что смена устойчивости синхронного режима происходит мягко, режима вынужденных колебаний - жестко.
3.1.2.  Структура плоскости параметров (£,е)
Более полное представление о процессах синхронизации жесткого автоколебательного режима дает структура плоскости параметров (£, £ укороченной системы уравнений (рис. 14) и соответствующие ей фазовые портреты (рис. 15 и рис. 16). Часть разбиения на рис. 14 унаследована от структуры
27
плоскости параметров (£,е) при синхронизации мягкого режима (рис. 5), в этой части обозначения бифуркационных кривых и областей сохранено. Далее остановимся на описании новых бифуркационных кривых и областей.
• пунктирная линия - бифуркация двукратного состояния равновесия. Она удовлетворяет условию А(£,е) = 0. Эта линия ограничивает область существований состояний равновесия: O4 и O5 (рис. 15а). На верхней границе этой области состояния равновесия O4 и O5 сливаются и исчезают, на нижней - исчезают O4 и O3;
• линия 6 - бифуркация Андронова-Хопфа, смена устойчивости состояния равновесия O5. На этой линии выполняется условие а(£, е)=0. Смена устойчивости O5 происходит при увеличении е. Здесь первая ляпунов-ская величина положительна, смена устойчивости O5 происходит в результате стягивания в точку неустойчивого колебательного предельного цикла S\'7
• линия 7 - бифуркация петли сепаратрис седла O4. Пересечение этой кривой с ростом е или £ приводит к рождению цикла Si, седловая величина на линии 7 положительна. Концевыми точками для бифуркационной кривой служат точки d и е, которые аналогичны точкам а и b . В точке d выполняются условия теоремы Богданова, в этой точке собираются вместе линия 6 и линия 7, на пунктирной линии слева от точки d двукратное состояние равновесия имеет устойчивую узловую часть, справа - неустойчивую. В точке е петля сепаратрис седла превращается в петлю сепаратрис седло-узла;
• линия 8 - бифуркация петли сепаратрис седло-узла, петля не охватывает цилиндр V +. Из этой петли рождается неустойчивый колебательный предельный цикл Si. Линия 8 есть участок пунктирной линии, расположенный между точками ей/;
• линия 9 - бифуркация петли сепаратрис седло-узла, петля охватывает цилиндр V + . Из этой петли рождается неустойчивый вращательный предельный цикл S2. Линия 9 есть участок пунктирной линии, расположенный ниже точки /;
• линия 10 разделяет области существования колебательного S1 и вращательного S2 предельных циклов.
Новые бифуркационные кривые приводят к появлению на плоскости е) новых областей D7 — D11. Соответствующие этим областям фазовые портреты представлены на рис. 16. В области D0 структуру фазового пространства
28
Рис. 14. Структура плоскости параметров (£, е) модели (6) в случае жесткого автоколебательного режима
(в)
(г)
Рис. 15. Фазовые портреты укороченной модели генератора в случае синхронизации жесткого автоколебательного режима (начало)
29
(ж) (з)
Рис. 16. Фазовые портреты укороченной модели генератора в случае синхронизации жесткого автоколебательного режима (продолжение)
30
определяют (рис. 15а): устойчивые состояния равновесияO1 и O5, неустойчивый узел (фокус) O3, седлa O2 и O4. При значениях параметров из области D0 система демонстрирует бистабильное поведение. В зависимости от начальных условий фазовые траектории устремляются либо к состоянию равновесия Oi, соответствующему режиму синхронизации, либо к состоянию равновесия O5 - режиму вынужденных колебаний. Границами бассейнов притяжения n(Oi) и n(O5) аттракторов O1 ж O5 служат входящие сепаратрисы седла O4.
В области D7 на фазовой плоскости V + располагаются неустойчивые состояния равновесия O4, O3, O5 и вращательный устойчивый предельный цикл L2 (рис. 16г). Цикл L2 является единственным аттрактором динамической модели, независимо от начальных условий в системе реализуется режим биении.
В области D8 система демонстрирует бистабильное поведение (рис. 16д). Здесь, если начальные условия принадлежат области n(O5) ( на рис. 15в область n(O5) выделена серым цветом), то устанавливается режим вынужденных колебаний (устойчивое состояние равновесия O5), в противном случае устанавливается режим биений (цикл L2). Разграничивает области притяжения n(O5) и n(L2) неустойчивый колебательный предельный цикл Si.
В областях D9, D10, D11 поведение генератора аналогично поведению в Dg. Переходы из области D8 в области D9,D10 ж D11 связаны с бифуркациями неустойчивых траекторий, которые приводят к перераспределению фазовых потоков, т.е. изменению границ областей притяжения n(O5) и n(L2). В области D9 бассейны притяжения n(O5) и n(L2) разделяют входящие сепаратрисы седла O4 (рис. 16е), в области D10 - неустойчивый колебательный предельный цикл S1 (рис. 16ж), в области D11 - неустойчивый вращательный предельный цикл S2 (рис. 1бз). Области с бистабильным поведением на рис. 14 выделены серым цветом.
3.1.3.  Особенности синхронизации жесткого режима
Особенности синхронизации жесткого режима связаны с наличием у автономного генератора двух устойчивых режимов: автоколебательного и невозбужденного генератора. Если генератор изначально находится в автоколебательном режиме, то сценарии синхронизации внешним сигналом не отличаются от сценариев синхронизации мягкого режима. Если подействовать внешней силой на невозбужденный генератор, то динамика неавтономного генератора становится более разнообразной. Сценарии динамического поведения определяются начальными значениями амплитуды и частоты внешнего воздействия, а так же путями их дальнейшего изменения. Все сценарии для рассматриваемого генератора могут быть выявлены из совокупного рас-
31
смотрения резонансных кривых (рис. 13в), пространства параметров (рис. 14) и фазовых портретов (рис. 15 и рис. 16). Мы ограничимся анализом лишь нескольких возможных путей развития динамических режимов неавтономного генератора при вариации £ для различных значений е.
Пусть автономный генератор не возбужден. При подаче внешнего сигнала амплитуды e=ei или е=е2 при изменении расстройки £ получим просто резонансную кривую вынужденных колебаний. На рис. 13в этому режиму отвечают нижние линии 1 и 2, которые отражают эволюцию координаты р5 устойчивого состояния равновесия 05($>§, pt) (см. рис. 15,16). При действии на невозбужденный генератор сигналом амплитуды е чуть больше е3 при больших расстройках по-прежнему будем иметь вынужденные колебания, амплитуда которых растет с уменьшением £ ( см. рис. 13в), с приходом на границу А = 0 (при £ = £о) области седел (нижний эллипс) этот режим разрушается. На рис. 14 разрушение режима происходит при переходе из области D0 в Dl7 а на фазовом портрете - исчезновением состояния равновесия 05. Однако при этой же расстройке £0 существует устойчивая ветвь АЧХ вверху (эту ветвь определяет координата р\ устойчивого состояния равновесия 01(ф\, р\)): амплитуда колебаний на частоте внешней силы скачком возрастает - генератор возбуждается и сразу попадает в режим синхронизации. Дальнейшее уменьшение расстройки £ до нуля ведет к росту амплитуды синхронных колебаний. Если же после установления синхронного режима начать увеличивать расстройку £, то амплитуда синхронных колебаний уменьшается (см. рис. 13в), при переходе через границу А = 0 области седел (верхний эллипс) синхронный режим сменяется режимом биений. На рис. 14 смена режима обусловлена переходом из области D1 в область D6, в фазовом пространстве - исчезновением состояний равновесия 01 и рождением устойчивого предельного цикла L2. При дальнейшем увеличении £ цикл L2 (режим биений) сохраняется.
При большой амплитуде е, существующие при больших расстройках вынужденные колебания с уменьшением £ переходят в модулированные колебания (биения). На рис. 13в этот процесс отражает нижняя часть линии 5, которая при уменьшении £ пересекает границу устойчивости а = 0, L1 > 0. На плоскости параметров (£,е) смена режима определяется выходом из области D10 в D6, в фазовом пространстве - жесткой сменой устойчивости состояния равновесия 05 и переходом на предельный цикл L2. Режим биений существует в диапазоне £2  < £ < £1, где £1  и £2  - значения  £ в точках пересечения
резонансной кривой границ а = 0,L1 > 0 и а = 0,L1 < 0 соответственно. Заметим, что при уменьшении £ от £1 до £2 режим биений претерпевает качественные изменения, обусловленные сменой типа предельного цикла с второго рода на первый (вращательный цикла L2 сменяется колебательным
32
циклом L\). На плоскости параметров (£,ф эти изменения происходят на границе областей D6 и D5 При £ = £2 в фазовом пространстве устойчивый цикл Li стягивается в точку Оц состояние равновесия Oi (ф1, р\) приобретает устойчивость и становится образом синхронных колебаний. В неавтономном генераторе устанавливается режим синхронизации.
При дальнейшем уменьшении расстройки £ амплитуда синхронных колебаний возрастает (см. линию 5 на рис. 13в), при £ = 0 она принимает максимальное значение. При уходе расстройки в другую сторону (£ < 0) амплитуда синхронных колебаний убывает, при £ = -£2 наступает режим биений, который при £ < -£1 не исчезает (генератор остается возбужденным даже если выключить внешнюю силу). Аналогичная картина наблюдается, если от £ = 0 двигаться в сторону увеличения £ - при £ > £2 наступает режим биений, который сохраняется при £ > ф.
\section{Вопросы ж задания для самоконтроля}
  1.  В чем суть явления синхронизации?
  2.  Получить уравнения, описывающие поведение лампового генератора.
  3.  Какими динамическими уравнениями описывается явление синхронизации лампового генератора внешней силой? (Уравнения (4) - (6) относятся к классу линейных/нелинейных, автономных/неавтономных динамических систем. Какова размерность фазового пространства моделей (4)
  (б), какие фазовые переменные?)
  4.  Когда при изучении динамики модели (4) можно использовать метод Ван-дер-Поля?
  5.  В чем суть метода метода Ван-дер-Поля?
  6.  Найти условие самовозбуждения автономного (без внешнего воздействия) генератора.
  7.  От чего зависит амплитуда колебаний автономного генератора? Привести аналитические выражения, определяющие зависимость амплитуды колебаний от параметров схемы.
  8.  От чего зависит частота колебаний автономного генератора? При каких условиях частота генератора совпадает с частотой колебательного контура?
  33
  9.  Вычислить поправку к частоте колебаний автономного генератора при аппроксимации нелинейности лампы полиномом третьей (пятой) степени.
  10. Какие особые фазовые траектории модели (5) являются образами режима синхронизации, режима биений?
  11. Какие особые фазовые траектории модели (6) являются образами режима синхронизации, режима биений?
  12. Когда внешний сигнал считается слабым, когда сильным?
  13. Как изменяется структура фазового портрета модели (6) при выходе из полосы синхронизации в случае слабого/сильного сигналов?
  14. Дайте определение полосы синхронизации (удержания). Какими бифуркациями моделей (6) и (5) определяется граница полосы синхронизации в случае слабого и сильного сигналов?
  15. Дайте определение полосы захвата в режим синхронизации.
  16. В чем разница между полосой захвата в синхронный режим и полосой удержания синхронного режима (полосой синхронизации). Какая из полос больше? Могут ли полосы удержания и захвата совпадать?
  17. Когда полосы удержания и захвата совпадают (не совпадают)?
  18. Как зависит полоса синхронизации (захвата) от амплитуды внешнего сигнала?
  19. Как устанавливаются биения в случае слабого и сильного сигналов?
  20. Чем отличается мягкий режим возбуждения от жесткого?
  21. Как от мягкого режима возбуждения перейти к жесткому?
  22. Как найти состояния равновесия модели (6) и определить его тип?
  23. Когда периодическая траектория модели (5) устойчива.
  24. Чем отличается явление захватывания при жестком режиме возбуждения от явления захватывания при мягком режиме возбуждения.
  34
\section{Темы курсовых ж выпускных квалификационных работ}
  1.  Моделирование динамики неавтономной модели лампового генератора в жестком режиме возбуждения.
  2.  Анализ динамики лампового генератора при внешнем гармоническом воздействии в потенциально-автоколебательном режиме [15].
  3.  Моделирование динамики генератора с характеристикой лампы несимметричной относительно рабочей точки (например, для ламп П-7 и М-28, см. [4], Дополнение, § 1. Электронная лампа, стр.549).
\section{Экспериментальное исследование синхронизации}
(лабораторная работа)
\subsection{Описание установки}
  На  рис. 17 приведены общий вид и схема лабораторной установки для изучения явления вынужденной синхронизации. Установка состоит из: синхронизируемого автогенератора с изменяемым коэффициентом возбуждения
  (1) , генератора внешней силы регулируемой частоты и амплитуды (2), осциллографа (3).
\subsection{Задание 1. Изучение явления захватывания при мягком режиме возбуждения автогенератора}
  1.  Установить мягкий режим автономного генератора путем подбора напряжения на управляющей сетке лампы. Измерить амплитуду и частоту полученных автоколебаний.
  2.  Не меняя параметров схемы, подать внешнее воздействие. Снять зависимость амплитуды колебаний от частоты внешнего сигнала (АЧХ) для различных амплитуд внешнего сигнала. АЧХ снимается только для синхронного режима, вид которого представлен на рис. 18а,б.
  3.  Снять зависимости значений левой и правой границ полосы синхронизации от амплитуды внешнего сигнала при фиксированных значениях параметров автономного генератора (определить интервал значений частоты внешней силы, на границах которого реализуются синхронный ре-
  35
  Рис. 17. Общий вид (а) и схема (б) лабораторной установки для исследования явления вынужденной синхронизации
  жим (рис. 18а,б) и режим биений рис. 18в,г)4. Рассчитать ширину полосы синхронизации, проанализировать зависимость ширины полосы синхронизации от амплитуды внешней силы.
  4.  Снять зависимости значений левой и правой границ полосы захвата в режим синхронизации от амплитуды внешней силы при фиксированной амплитуде автономного генератора5. Рассчитать ширину полосы захва-
  4 При определении границ полосы синхронизации (удержания) движение по параметру должно осуществляться от синхронного режима к режиму биений.
  5 При определении границ полосы захвата в синхронный режим движение по параметру должно осуществляться от режима биений к синхронному режиму.
  36
  Рис. 18. Фигуры Лиссажу синхронизируемого генератора и внешней ЭДС (а) и (в), осциллограмма колебаний автогенератора (б) и (г) в режиме синхронизации (верхний ряд) и биений (нижний ряд)
  та, проанализировать зависимость ширины полосы захвата в синхронный режим от амплитуды внешней силы.
  5.  Сравнить полосы синхронизации и захвата при нескольких фиксированных значениях амплитуды внешней силы.
  6.  Снять зависимости амплитуды колебаний в синхронном режиме на границе полосы синхронизации (левой или правой). Проанализировать полученную зависимость, определить участки границ со слабым и сильным сигналом. Зарисовать (сфотографировать) осциллограммы режима биений в окрестности границы полосы синхронизации для слабого и сильного сигналов. Сравнить огибающие осциллограмм режима биений для слабого и сильного сигналов.
\subsection{Задание 2. Изучение явления захватывания при жестком режиме возбуждения}
  1. Подобрать параметры автономного генератора так, чтобы осуществлялся жесткий режим возбуждения.
  37
  2. Для жесткого режима возбуждения снять семейство АЧХ для различных амплитуд внешнего сигнала. Отметить ветви, отвечающие резонансу и режиму синхронизации. Сравнить полученные АЧХ с АЧХ для мягкого режима возбуждения.
  38
  Список литературы
[1] Андронов А.А., Витт А.А. К математической теории захватывания // Журнал прикладной физики. 1930. Т. 7. Вып. 4. С. 3.
[2] Андронов А. А., Витт А.А., К теории захватывания Ван дер Поля // Андронов А.А. Собрание трудов. 1956. М.: АН СССР.
[3] Блакьер О. Анализ нелинейных систем. М.: Мир, 1969. 400 с.
[4] Андронов А.А., Витт А.А., Хайкин С.Э. Теория колебаний. М.: Наука. Гл. ред. физ-мат. литературы, 3-е изд. 1981. (1-е изд. - 1937г.).
[5] Андронов А.А., Витт А.А., Хайкин С.Э. Теория колебаний. М.: Наука. Гл. ред. физ-мат. литературы, 2-е изд. 1959. 568 с.
[6] Горяченко В.Д. Элементы теории колебаний. М.: Высшая школа, 2-е изд. перераб. и доп. 2001.
[7] Барбашин Е.А., Табуева В.А. Динамические системы с цилиндрическим фазовым пространством. М.: Наука, 1969.
[8] Пиковский А., Розенблюм М., Курте Ю. Синхронизация. Фундаментальное нелинейное явление. М.: «Техносфера», 2003.
[9] Шахгильдян В.В., Ляховкин А.А. Системы фазовой автоподстройки частоты. М.: Связь, 1972.
[10]  Рабинович М.И., Трубецков Д.И. Ведение в теорию колебаний и волн. М.: Наука. Гл. ред. физ-мат. литературы, 1984.
[11]  Матросов В.В. Динамика нелинейных систем. Программный комплекс для исследования нелинейных динамических систем с непрерывным временем. Н.Новгород. ННГУ. 2002.
[12]  Некоркин В.И. Лекции по основам теории колебаний. Учебное пособие. Нижний Новгород. Изд-во Нижегородского гос. ун-та. 2012.
[13]  Арнольд В.И., Афраймович В.С., Ильяшенко Ю.С., Шильников Л.П. Теория бифуркаций. Современные проблемы математики. Фундаментальные направления. Итоги науки и техники. 1986. Т.5. М.:ВИНИТИ АН СССР.
[14]  Тураев Д.В., Шильников А.Л., Шильников Л.П. Некоторые математические проблемы классической синхронизации // Труды школы «Нелинейные волны’ 2004». 2005. С. 426-449.
39
[15]  Королев В.И., Постников Л.В. К теории синхронизации генератора автоколебаний // Изв. вузов. Радиофизика. 1969. Т. 12. № 3.
40
Валерий Владимирович Матросов
ВЫНУЖДЕННАЯ СИНХРОНИЗАЦИЯ
Учебно-методическое пособие
Федеральное государственное бюджетное образовательное учреждение высшего профессионального образования «Нижегородский государственный университет им. Н.И. Лобачевского» 603950, г. Нижний Новгород, пр. Гагарина, 23.
Формат 60 х 84 1/16.
Бумага офсетная. Печать цифровая. Гарнитура Times.
Уел. печ. л. 2,5. Уч-изд. л.
Заказ № . Тираж 150 экз.
Отпечатано в Центре цифровой печати РИУ Нижегородского госуниверситета им. Н.И. Лобачевского 603950, г. Нижний Новгород, пр. Гагарина, 23.